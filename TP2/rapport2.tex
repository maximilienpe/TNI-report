\documentclass{scrreprt}
\usepackage{array}
\usepackage{graphicx}
\usepackage{listings}
\usepackage{underscore}
\usepackage[bookmarks=true]{hyperref}
\usepackage[utf8]{inputenc}
\usepackage{float}
\usepackage[french]{babel}
\hypersetup{
    bookmarks=false,    % show bookmarks bar
    pdftitle={rapport_TP2_Lambolez_Petit},    % title
    pdfauthor={Théodore Lambolez, Maximilien Petit},                     % author
    pdfsubject={TeX and LaTeX},                        % subject of the document
    pdfkeywords={TeX, LaTeX, graphics, images}, % list of keywords
    colorlinks=true,       % false: boxed links; true: colored links
    linkcolor=blue,       % color of internal links
    citecolor=black,       % color of links to bibliography
    filecolor=black,        % color of file links
    urlcolor=black,        % color of external links
    linktoc=page            % only page is linked
}
\def\myversion{1.0}
\date{}
%\title
\usepackage{hyperref}
\begin{document}
\begin{figure}
   \begin{minipage}[c]{.46\linewidth}
      \includegraphics[scale=0.3]{images/telecom.png}
   \end{minipage} \hfill
   \begin{minipage}[c]{.46\linewidth}
      \includegraphics[scale=1.9]{images/lorraine.jpg}
   \end{minipage}
\end{figure}
\begin{flushright}
    \rule{15cm}{5pt}
    \vskip1cm
\end{flushright}
\begin{center}
	\vspace{3cm}
	\fbox{
	\begin{minipage}{0.9\textwidth}
        	\Huge{
			\textbf{
			\begin{center}
				Rapport \\Travaux Pratiques 2
				\vspace{0.5cm}
			\end{center}
			}
		}
	\end{minipage}
	}
\end{center}
\begin{flushright}
        \vspace{5cm}
	\huge{
        \textbf{
	Ecrit par \\
	\vspace{0,875cm}
	\href{mailto:theodore.lambolez@telecomnancy.eu}{Théodore Lambolez} \\
	\href{mailto:maximilien.petit@telecomnancy.eu}{Maximilien Petit}\\
	}
	}
        \vspace{0,5cm}
        \large{
	\textbf{
	\today\\
	}	
	}
\end{flushright}

\tableofcontents

\chapter{Contrôle de la qualité des connecteurs}

Nous avons choisi de réaliser une binarisation avant tout autre opération pour mettre en évidence les broches 
du connecteur présent sur l'image initialement en niveau de gris. Le seuil a été choisi en faisant la moyenne 
des valeurs de seuils bas obtenus sur l'échantillon d'image Connect fourni, 
à l'aide de l'outil auto-threshold utilisant l'option de minimisation de la variance.

\begin{table}[!h]
        \begin{center}
                \begin{tabular}{|c|c|c|}
                   \hline
                   Nom image & seuil bas & seuil haut \\
                   \hline
                   Connect1.tif & 110  & 255 \\
                   \hline
                   Connect2.tif & 116 & 255  \\
                   \hline
		   Connect3.tif & 116 & 255 \\
                   \hline 
		   Connect4.tif & 128 & 255 \\
                   \hline 
		   Connect5.tif & 126 & 255 \\
                   \hline 
		   Connect6.tif & 119 & 255 \\
                   \hline 
		   Connect7.tif & 124 & 255 \\
                   \hline 
		   Connect8.tif & 130 & 255 \\
                   \hline 
		   Connect9.tif & 129 & 255 \\
                   \hline 
		   Connect10.tif & 124 & 255 \\
		   \hline
 		   Connect11.tif & 126 & 255 \\
                   \hline
		   Connect12.tif & 129 & 255 \\
                   \hline 
 	
                \end{tabular}
        \end{center}
        \caption{Comparaisons des seuils de binarisations auto avec l'option de minimisation de la variance}
\end{table}


La moyenne obtenue du seuil bas est de 123. On utilisera donc l'outil de binarisation morphologique en utilisant
le seuil 123. 


//1.3 Améliorations

Cette extension n'est pas forcément nécessaire pour une pièce à tester car on peut définir une zone dans laquelle la placer. Par contre, pour tester plusieurs capteurs en même temps 
elle est interessante si on localise les différentes broches. Elle est industriellement viable si les conditions dans lesquelles le programme a été écris sont respectées (notamment
la luminosité).

\chapter{Identification et mesures}

La résolution obtenue est de Pixel Width 0,5114.




\end{document}

